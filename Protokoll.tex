\documentclass[11pt]{article}
%Gummi|061|=)
\usepackage{amsmath}
\usepackage{amsthm}
\usepackage{amsbsy}
\usepackage{amssymb}
\usepackage{inputenc}
\usepackage{graphicx}
\usepackage{selinput}
\SelectInputMappings{%
adieresis={ä},
germandbls={ß},
}
\title{\textbf{Versuch}}
\author{Martin Bieker\\
		Julian Surmann\\
		\\
		Durchgef\"{u}hrt am 14.11.2013\\
		Tu Dortmund}
\date{}
\usepackage{graphicx}
\begin{document}
\renewcommand\tablename{Tabelle}
\renewcommand\figurename{Abbildung}
\maketitle
\thispagestyle{empty}
\newpage
\clearpage
\setcounter{page}{1}


\section{Einleitung}
Hier Einleitung einsetzen.
\section{Theorie}
Hier Theorie einsetzen, Formeln mit:
\begin{equation}
F O R M E L  
\end{equation}
\section{Aufbau und Durchf\"{u}hrung}
Hier folgt der Aufbau und die Durchführung.
\section{Auswertung}



\section{Diskussion}
Hier kommt die Diskussion hin.
\section{Literatur- und Abbildungsverzeichnis}
Hier befindet sich das Literatur- und Abbildungsverzeichnis.
\section{Anhang}
Hier stehen die im Anhang angefügten Dokumente.
\end{document}
