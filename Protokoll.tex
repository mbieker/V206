\documentclass[11pt]{article}
%Gummi|061|=)
\usepackage{amsmath}
\usepackage{amsthm}
\usepackage{amsbsy}
\usepackage{amssymb}
\usepackage{inputenc}
\usepackage{graphicx}
\usepackage{selinput}
\usepackage{here}
\SelectInputMappings{%
adieresis={ä},
germandbls={ß},
}
\title{\textbf{Versuch V 206: Wärmepumpe}}
\author{Martin Bieker\\
		Julian Surmann\\
		\\
		Durchgef\"{u}hrt am 14.11.2013\\
		Tu Dortmund}
\date{}
\usepackage{graphicx}
\begin{document}
\renewcommand\tablename{Tabelle}
\renewcommand\figurename{Abbildung}
\maketitle
\thispagestyle{empty}
\newpage
\clearpage
\setcounter{page}{1}


\section{Einleitung}
Im folgenden Versuch geht es um eine Wärmepumpe. Eine Wärmepumpe entzieht einer kälteren Umgebung thermische Energie und gibt diese an eine wärmere Umgebung ab. Aus dem zweiten Hauptsatz der Thermodynamik folgt, dass die Wärmepumpe dafür technische Arbeit verrichten muss.
\section{Theorie}
In den folgenden Ausführungen soll A die von der Wärmepumpe zu verrichtende Arbeit, $\nu$ die Güteziffer der Wärmepumpe und $Q$ eine Wärmemenge sein. Aus dem ersten Hauptsatz der Thermodynamik folgt also:
\begin{equation}
\label{F1}
Q_1 = Q_2 + A \Rightarrow \nu = \frac{Q_1}{A}
\end{equation}
Für infinitesimal kleine Temparaturänderungen folgt dann aus dem zweiten Hauptsatz der Thermodynamik:
\begin{equation}
\label{F2}
\frac{Q_1}{T_1} - \frac{Q_2}{T_2} = 0
\end{equation}
Dieser Zusammenhang gilt allerdings nur bei Prozessen, die reversibel sind. Diese idealisierte Form gilt bei einer Wärmepumpe nicht. Für die Wärmepumpe gilt
\begin{equation}
\label{F3}
\frac{Q_1}{T_1} - \frac{Q_2}{T_2} > 0.
\end{equation}
Wenn man Formel (\ref{F2}) nach $Q_2$in Formel (\ref{F1}) einsetzt, erhält man
\begin{equation}
\label{F4}
Q_1 = A + \frac{T_2}{T_1} Q_1
\end{equation}
und
\begin{equation}
\label{F5}
\nu_{id} = \frac{Q_1}{A} = \frac{T_1}{T_1-T_2}.
\end{equation}
Für die reale Wärmepumpe gilt dann, aus Formel (\ref{F1}) und (\ref{F3}) folgend,
\begin{equation}
\label{F6}
\nu_{real} < \frac{T_1}{T_1-T_2}
\end{equation}
Für eine umfassende Auswertung des Versuches sind einige weitere Ausdrücke von Nöten.
So gilt für die mechanische Kompressorleistung
\begin{equation}
\label{F7}
N_{mech} = \frac{\Delta A_m}{\Delta t} = \frac{1}{k-1} \left(p_b \sqrt[k]{\frac{p_a}{p_b}}-p_a \right) \frac{1}{\rho} \frac{\Delta m}{\Delta t}.
\end{equation}
Die im Versuch pro Zeiteinheit gewonnene Wärmemenge ist
\begin{equation}
\label{F8}
\frac{\Delta Q_2}{\Delta t} = (m_1c_w + m_kc_k) \frac{\Delta T_1}{\Delta t}.
\end{equation}
Für die Güteziffer der Wärmepumpe ergibt sich dann
\begin{equation}
\label{F9}
\nu_{real} = \frac{\Delta Q_1}{\Delta t N} = \frac{(m_1c_w + m_kc_k) \frac{\Delta T_1}{\Delta t}}{N}.
\end{equation}
Für die Bestimmung des Massendurchsatzes gilt
\begin{equation}
\label{F10}
\frac{\Delta m}{\Delta t} = \frac{(m_2 c_w + m_k c_w) \frac{\Delta T_2}{\Delta t}}{L}
\end{equation}

\section{Aufbau und Durchf\"{u}hrung}
Hier folgt der Aufbau und die Durchführung.
\section{Auswertung}
Im Folgenden sollen einige f\"ur eine W\"armepumpe chrakteritischen Gr\"o\ss en bestimmt werden. Die w\"ahrend der Versuchsdurchf\"uhrung gewonnenen Messwerte sind im Anhang in Tabelle \ref{rohdaten} zu finden. Ausgleichs- und Fehlerrechnungen in der Auswertung wurden mit verschiedenen Pythonbibliotheken (SciPy, uncertainties und matplotlib) durchgef\"uhrt. Dabei verwendet Python uncertainties die Fehlerfortpflanzung nach Gau\ss. Bei diesem Verfahren ist der Fehler einer von mehreren Variablen mit den Fehleren $\Delta x_i$ abh\"angeden Gr\"o\ss e $y$ durch 
\begin{equation}
\Delta y = \sqrt{\sum_i^N (\frac{\partial y}{\partial x_i} \cdot \Delta x_i)^2}
\end{equation}
gegeben. 
\subsection{Interpolation der Daten}
Die aufgenommen Temperaturwerte $T_1$ und $T_2$ sollen durch eine quadratische Funktion
\begin{equation}
T(t) = At^2 + Bt + C
\end{equation}
approximiert werden. Die  nicht lineare Ausgleichsrechnung ergibt f\"uer die Paramater $A, B, C$:
\begin{table}[H]
\centering
\begin{tabular}{c|c|c|c}

& $A [\frac{K}{s^2}]$ & $B [\frac{K}{s}]$ & $C [K]$ \\
\hline
$T_2$& $(-3.5\pm0.1)10^{-6}$ & $0.0222\pm0.0002$ &$ 293.68\pm0.08$\\
$T_1$ & $(3.9\pm0.1)10^{-6}$ & $-0.0174\pm0.0002$ & $294.89\pm0.09$\\
\end{tabular}
\caption{Regressions-Paramter}
\end{table} \noindent
Die  G\"utezahlen $\nu_{real}$ und $\nu_{id}$ sowie der Massendurchsatz und die Leistung des Kompressors werden f\"ur vier gleich\"ma\ss ig \"uber den Messzeitraum verteilte Zeiten gemessen. Dazu werden zun\"achst die Diffrentialquotienten $\frac{d T_1}{\partial t}$ und $\frac{dT_2}{\partial t}$ bestimmt. Dazu werden die Approximationsgleichungen abgeleitet. 
\begin{equation}
\frac{dT}{dt}(t) = 2At + B
\end{equation}
Daraus ergibt mit den Paramterwerten und den Zeiten $t$:
\begin{table}[H]
\centering 
\begin{tabular}{c|c|c|c|c}
\hline
$t [s]$ & $T_1 [K]$ & $\frac{dT_1}{dt} \frac{K}{s}]$ & $T_2 [K]$ & $\frac{dT_2}{dt} [\frac Ks]$  \\
\hline
480 & 303.65 & 0.01889+/-0.00022 & 287.35 & -0.01370+/-0.00024 \\
960 & 311.85 & 0.01556+/-0.00028 & 281.65 & -0.00998+/-0.00030 \\
1440 & 318.45 & 0.01223+/-0.00035 & 277.85 & -0.0063+/-0.0004 \\
1920 & 323.65 & 0.0089+/-0.0004 & 275.65 & -0.0026+/-0.0005 \\
\hline
\end{tabular}
\end{table}
\section{Diskussion}
Hier kommt die Diskussion hin.
\section{Literatur- und Abbildungsverzeichnis}
Hier befindet sich das Literatur- und Abbildungsverzeichnis.
\section{Anhang}
Hier stehen die im Anhang angefügten Dokumente.
\end{document}
