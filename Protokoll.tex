\documentclass[11pt]{article}
%Gummi|061|=)
\usepackage{amsmath}
\usepackage{amsthm}
\usepackage{amsbsy}
\usepackage{amssymb}
\usepackage{inputenc}
\usepackage{graphicx}
\usepackage{selinput}
\SelectInputMappings{%
adieresis={ä},
germandbls={ß},
}
\title{\textbf{Versuch V 206: Wärmepumpe}}
\author{Martin Bieker\\
		Julian Surmann\\
		\\
		Durchgef\"{u}hrt am 14.11.2013\\
		Tu Dortmund}
\date{}
\usepackage{graphicx}
\begin{document}
\renewcommand\tablename{Tabelle}
\renewcommand\figurename{Abbildung}
\maketitle
\thispagestyle{empty}
\newpage
\clearpage
\setcounter{page}{1}


\section{Einleitung}
Im folgenden Versuch geht es um eine Wärmepumpe. Eine Wärmepumpe entzieht einer kälteren Umgebung thermische Energie und gibt diese an eine wärmere Umgebung ab. Aus dem zweiten Hauptsatz der Thermodynamik folgt, dass die Wärmepumpe dafür technische Arbeit verrichten muss.
\section{Theorie}
In den folgenden Ausführungen soll A die von der Wärmepumpe zu verrichtende Arbeit, $\nu$ die Güteziffer der Wärmepumpe und $Q$ eine Wärmemenge sein. Aus dem ersten Hauptsatz der Thermodynamik folgt also:
\begin{equation}
\label{F1}
Q_1 = Q_2 + A \Rightarrow \nu = \frac{Q_1}{A}
\end{equation}
Für infinitesimal kleine Temparaturänderungen folgt dann aus dem zweiten Hauptsatz der Thermodynamik:
\begin{equation}
\label{F2}
\frac{Q_1}{T_1} - \frac{Q_2}{T_2} = 0
\end{equation}
Dieser Zusammenhang gilt allerdings nur bei Prozessen, die reversibel sind. Diese idealisierte Form gilt bei einer Wärmepumpe nicht. Für die Wärmepumpe gilt
\begin{equation}
\label{F3}
\frac{Q_1}{T_1} - \frac{Q_2}{T_2} > 0.
\end{equation}
Wenn man Formel (\ref{F2}) nach $Q_2$in Formel (\ref{F1}) einsetzt, erhält man
\begin{equation}
\label{F4}
Q_1 = A + \frac{T_2}{T_1} Q_1
\end{equation}
und
\begin{equation}
\label{F5}
\nu_{id} = \frac{Q_1}{A} = \frac{T_1}{T_1-T_2}.
\end{equation}
Für die reale Wärmepumpe gilt dann, aus Formel (\ref{F1}) und (\ref{F3}) folgend,
\begin{equation}
\label{F6}
\nu_{real} < \frac{T_1}{T_1-T_2}
\end{equation}
Für eine umfassende Auswertung des Versuches sind einige weitere Ausdrücke von Nöten.
So gilt für die mechanische Kompressorleistung
\begin{equation}
\label{F7}
N_{mech} = \frac{\Delta A_m}{\Delta t} = \frac{1}{k-1} \left(p_b \sqrt[k]{\frac{p_a}{p_b}}-p_a \right) \frac{1}{\rho} \frac{\Delta m}{\Delta t}.
\end{equation}
Die im Versuch pro Zeiteinheit gewonnene Wärmemenge ist
\begin{equation}
\label{F8}

\end{equation}



\section{Aufbau und Durchf\"{u}hrung}
Hier folgt der Aufbau und die Durchführung.
\section{Auswertung}
\subsection{Ausgleichsrechnung}
Abbildung \ref{temp} zeigt den Zeitlichen Verlauf der Temperaturen $T_a$ und $T_b$. Diese Messwerte werden nun jeweils durch eine nicht-lineare Ausgleichsrechnung durch ein Polynom zweiter Ordnung
 \begin{equation}
T(t)= At^2+Bt+C 
 \end{equation}
 approximiert. Die mit Python durchgeführte Rechnung ergibt für die Koeffizienten $A, B, C$:
\begin{table}[h]
\centering

\begin{tabular}{l|r|r|r}
	 & $A [^\circ Cs^{-2}]$ & $B [^\circ Cs^{-1}]$ & $C [^\circ C]$ \\
\hline
$T_1$ & $(-3.47\pm0.10)10^{-6}$ & $0.02223\pm0.00020$ & $20.53\pm0.08$\\
$T_2$ & $(3.87\pm0.11)10^{-6}$ & $-0.01741\pm0.00022$ & $21.74\pm0.09$\\
\end{tabular}
\end{table}
Zur Bestimmung der Differentialkoeffizienten werden die Näherungslösungen abgeleitet:
\begin{equation}
\frac{dT}{dt} = 2At+B
\end{equation}
Mit dieser Gleichung wurden die Differentialquotitenten der Funktionen $T_1$ und $T_2$ an fünf gleichmäßig über das Messintervall verteilten Zeitpunkten $t$ ermittelt (Tabelle \ref{diffqout}).
\begin{table}[h]
\centering
\begin{tabular}{c|c|c}
$t [s]$ & $\frac{dT_1}{dt} [^\circ Cs^{-1}]$ & $\frac{dT_1}{dt}[^\circ Cs^{-1}]$\\
\hline
360  &$0.0218\pm0.0002$ & $-0.0169\pm0.0002$\\
720  &$0.0214\pm0.0002$ & $-0.0165\pm0.0002$\\
1080 &$0.0210\pm0.0002$ & $-0.0160\pm0.0002$\\
1440 &$0.0206\pm0.0002$ & $-0.0155\pm0.0002$\\
1800 &$0.0202\pm0.0002$ & $-0.0151\pm0.0002$\\
\end{tabular}
\label{diffquot}
\caption{Differentialquoutienten von $T_1$ und $T_2$ für verschiedene Zeiten}
\end{table}
\begin{table}
\begin{tabular}{r|r|r|r|r|r}
$t [s]$ & $p_a [10^5Pa]$ & $p_b [10^5Pa]$ & $T_1 [^\circ]$ & $T_2 [^\circ]$ & P [W] \\
\hline
0.0 & 4.0 & 3.6 & 21.1 & 21.0 & 0.0\\
1.0 & 3.21 & 5.1 & 21.9 & 21.0 & 115.0\\
2.0 & 3.4 & 5.4 & 23.0 & 20.1 & 120.0\\
3.0 & 3.55 & 5.6 & 24.1 & 19.0 & 123.0\\
4.0 & 3.6 & 5.8 & 25.3 & 17.9 & 125.0\\
5.0 & 3.6 & 6.0 & 26.6 & 16.8 & 126.0\\
6.0 & 3.55 & 6.3 & 28.0 & 15.9 & 125.0\\
7.0 & 3.4 & 6.6 & 29.3 & 15.0 & 125.0\\
8.0 & 3.3 & 6.8 & 30.5 & 14.2 & 123.0\\
9.0 & 3.2 & 7.0 & 31.6 & 13.5 & 122.0\\
10.0 & 3.1 & 7.2 & 32.7 & 12.7 & 121.0\\
11.0 & 3.05 & 7.4 & 33.8 & 12.0 & 121.0\\
12.0 & 3.0 & 7.7 & 34.8 & 11.2 & 121.0\\
13.0 & 2.9 & 7.9 & 35.8 & 10.5 & 121.0\\
14.0 & 2.85 & 8.1 & 36.9 & 9.8 & 121.0\\
15.0 & 2.8 & 8.3 & 37.7 & 9.2 & 121.0\\
16.0 & 2.75 & 8.5 & 38.7 & 8.5 & 122.0\\
17.0 & 2.7 & 8.8 & 39.8 & 7.8 & 123.0\\
18.0 & 2.65 & 9.0 & 40.5 & 7.4 & 123.0\\
19.0 & 2.6 & 9.2 & 41.4 & 6.9 & 124.0\\
20.0 & 2.55 & 9.3 & 42.2 & 6.4 & 124.0\\
21.0 & 2.55 & 9.5 & 42.9 & 6.0 & 125.0\\
22.0 & 2.5 & 9.7 & 43.8 & 5.5 & 125.0\\
23.0 & 2.5 & 9.9 & 44.5 & 5.1 & 125.0\\
24.0 & 2.45 & 10.1 & 45.3 & 4.7 & 125.0\\
25.0 & 2.45 & 10.3 & 46.0 & 4.4 & 125.0\\
26.0 & 2.4 & 10.4 & 46.7 & 4.0 & 125.0\\
27.0 & 2.4 & 10.6 & 47.4 & 3.7 & 125.0\\
28.0 & 2.4 & 10.7 & 48.0 & 3.5 & 125.0\\
29.0 & 2.4 & 10.9 & 48.7 & 3.2 & 125.0\\
30.0 & 2.39 & 11.1 & 49.3 & 2.9 & 125.0\\
31.0 & 2.38 & 11.3 & 49.9 & 2.7 & 125.0\\
32.0 & 2.37 & 11.3 & 50.5 & 2.5 & 125.0\\
\end{tabular}
\centering
\caption{Ergebnisse der Messung}
\label{data}
\end{table}

\section{Diskussion}
Hier kommt die Diskussion hin.
\section{Literatur- und Abbildungsverzeichnis}
Hier befindet sich das Literatur- und Abbildungsverzeichnis.
\section{Anhang}
Hier stehen die im Anhang angefügten Dokumente.
\end{document}
